
\chapter{Ethical Considerations}
\vspace{-2ex}
This project mainly focuses on mining news articles to visualise what can be interpreted as the key news and entities in an industry. From an intelligence standpoint, this could be extremely useful as it condenses large amounts of news to give a visual overview of an industry. Though this project might highlight recent developments and trends in a sector, it is extremely unlikely to influence and propagate some radicalised opinion. 

There is no direct human interaction in this project, neither does it require any collection or direct use of personal data. It should be mentioned that data will be used from large open-source knowledge graphs where there may be data regarding a person of interest that may be deemed personal. E.g. Person: Donald Trump, Location: White house. However, any data that is in public domain will not be considered private or sensitive. Wikidata releases all its data under Creative Commons CC0 (public domain). \hyperlink{35}{[35]} 

One important ethical consideration, however, is to think about any bias introduced in the model/ visualisation due to the nature of the input data which inherently might contain some underlying bias. For example, if the data used for the energy industry primarily contains a majority of right-wing articles talking about climate change being a hoax and refusing to support clean energy alternatives, then that will be reflected in the visualisation through sentiment and topics extracted. This may result in a negative sentiment associated with clean energy. In order to combat this to the best of my degree, I will need to use a variety of different sources that balance the bias. However, it is important to note that most news in its nature can be polarising and biased, so therefore it can be hard to gauge what quantifies as unbiased.

Another key ethical consideration is that of using software and data that may have copyright licensing implications. A key aspect of this project is to use news articles for entity, sentiment and topic extraction. The news articles can be obtained from various online sources of major news companies such as BBC, The Guardian etc. Using the articles (which are an Intellectual Property (IP) of these companies) for "non-commercial" data analysis is not a copyright infringement as the copyright law has been updated to provide an exception for "Text and data mining technologies to help researchers process large amounts of data" \hyperlink{36}{[36]}. All other libraries and software packages used such as Dash, AllenNLP are open-source and free to use for the purposes of this project (at least for academic use).